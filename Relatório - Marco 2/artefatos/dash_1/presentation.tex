Este artefato corresponde ao Dashboard Estratégico, destinado ao apoio à tomada de decisão da diretoria da instituição, com foco em indicadores consolidados de desempenho, impacto institucional e eficiência das atividades de extensão universitária.

O dashboard apresenta uma visão analítica, agregada e histórica, permitindo a identificação de tendências e padrões relevantes para o planejamento estratégico. Por não estar associado a operações em tempo real, suas informações são consumidas de forma pontual, em contextos como reuniões gerenciais, avaliações institucionais e definição de políticas de extensão.
O painel é composto por quatro gráficos analíticos estratégicos, cada um derivado de uma visão materializada específica, conforme descrito a seguir:
\begin{enumerate}
    \item Desempenho mensal das atividades, apresentando o total de participações, certificados emitidos e a taxa percentual de certificação ao longo do tempo.
    \item Distribuição de participantes por área de estudo e gênero, permitindo a análise de diversidade e aderência institucional das atividades ofertadas.
    \item Impacto das parcerias institucionais, evidenciando o alcance das ações de extensão a partir da categoria dos parceiros envolvidos.
    \item Evolução mensal da carga horária consumida e da participação, possibilitando a avaliação da eficiência no uso dos recursos acadêmicos.
\end{enumerate}
Para garantir desempenho adequado na recuperação desses indicadores, optou-se pela utilização de Visões Materializadas, responsáveis por armazenar previamente os resultados das agregações complexas. A atualização dessas estruturas é realizada de forma controlada por meio de \textit{Stored Procedures}, assegurando consistência dos dados sem impactar as operações transacionais do sistema \cite{elmasri_sistemas_2019}.