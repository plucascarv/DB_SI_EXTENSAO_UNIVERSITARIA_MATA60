Este artefato compreende o back-end de dados destinado ao suporte da gestão intermédia e coordenação pedagógica. Enquanto o dashboard estratégico foca em indicadores macroscópicos, este painel operacional necessita de uma granularidade superior para permitir o monitoramento tático das atividades de extensão. O objetivo é fornecer aos gestores uma visão longitudinal sobre a evolução do programa, a diversidade da oferta académica e o perfil institucional dos participantes.

Para garantir a fluidez na navegação e a responsividade dos seis componentes gráficos exigidos, a arquitetura manteve o padrão de utilização de Materialized Views. Esta decisão técnica justifica-se pela necessidade de realizar segregações históricas (agrupamentos por ano e tipo) sobre a totalidade da base de dados. A execução destas consultas analíticas em tempo real sobre tabelas transacionais normalizadas acarretaria latência perceptível e bloqueios de leitura, prejudicando a operação de cadastro.
O conjunto de dados foi estruturado em seis visões materializadas que alimentam as seguintes dimensões de análise:
\begin{enumerate}
    \item Evolução e Qualidade (Séries Temporais):
    \begin{itemize}
        \item Uma visão dedicada à Evolução Anual da Participação, que confronta o número total de inscrições com o número de participantes únicos, permitindo medir a fidelização e a recorrência dos alunos ao longo dos anos.
        \item Uma visão de Taxa de Certificação, que monitoriza a eficácia pedagógica ano a ano, calculando o percentual de concluintes face aos inscritos. Este indicador é crucial para identificar quedas bruscas de rendimento ou evasão em períodos específicos.
    \end{itemize}
    
    \item Segmentação da Oferta:
    \begin{itemize}
        \item Uma matriz analítica de Histórico por Área de Estudo, que permite aos coordenadores visualizar quais as disciplinas (ex: Tecnologia, Saúde, Humanas) que apresentam tendências de crescimento ou declínio na procura.
        \item Uma análise de Distribuição por Tipo de Atividade, que segrega os dados por tipologia (Workshops, Palestras, Cursos), auxiliando no planeamento do \textit{mix} de produtos educacionais a ser ofertado no próximo ciclo.
    \end{itemize}
    
    \item Perfil e Benchmarking:
    \begin{itemize}
        \item Uma visão de Impacto Institucional, que mapeia a origem dos participantes agrupando-os pela instituição declarada. Este componente permite entender a penetração do programa noutras instituições de ensino e a diversidade do público atendido.
        \item Um relatório de Média Histórica por Atividade, que funciona como um ranking de popularidade, calculando a média de público de cada evento específico. Isso permite comparar o desempenho de uma edição atual contra a sua própria média histórica.
    \end{itemize}
\end{enumerate}

Para a manutenção destes dados, foi implementada uma Stored Procedure de Orquestração (sp\_refresh\_dashboard2). Diferente das atualizações individuais, esta rotina unificada executa o refresh sequencial de todas as seis visões materializadas numa única transação lógica, simplificando o agendamento de tarefas de manutenção pelo administrador da base de dados.
A implementação deste artefato cumpre rigorosamente os requisitos de preservação e performance. Ao pré-calcular estatísticas anuais e médias históricas, o sistema evita o recálculo redundante de dados passados que raramente sofrem alteração. Adicionalmente, a privacidade é assegurada, pois os gestores operacionais acedem apenas a dados estatísticos consolidados por instituição ou ano, sem exposição direta aos registos individuais dos alunos nas tabelas transacionais.