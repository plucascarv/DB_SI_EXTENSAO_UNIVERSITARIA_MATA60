Este artefato representa a interface primária para a entrada de novas atividades de extensão no sistema. O objetivo é prover ao gestor um panorama estatístico imediato sobre as áreas de estudo antes que a operação de cadastro seja efetivada, permitindo um planejamento mais estratégico da oferta de cursos e eventos. Para viabilizar essa funcionalidade, o back-end foi estruturado em dois componentes distintos de banco de dados: uma camada analítica de leitura e uma camada transacional de escrita.

Na camada analítica, optou-se pela criação de uma Visão Materializada. Diferentemente de uma view padrão, que recalcularia as métricas a cada acesso, a visão materializada armazena fisicamente o resultado de agregações complexas envolvendo múltiplas tabelas \cite{silberschatz_database_2019}. Esta estrutura foi desenhada para calcular indicadores de qualidade e não apenas de volume, apresentando a média de alunos por turma, a taxa percentual de certificação e o tempo de ociosidade de cada área. Para garantir que a recuperação desses dados analíticos seja instantânea durante o carregamento da tela, foi criado um índice do tipo B-Tree sobre a coluna de agrupamento, eliminando a necessidade de varreduras sequenciais no disco.

Na camada transacional, a lógica de persistência foi encapsulada em uma Stored Procedure. Esta abordagem foi escolhida para garantir a atomicidade e a integridade da operação. A rotina recebe todos os dados necessários e executa um fluxo sequencial rigoroso: primeiramente, valida a existência prévia do identificador da atividade, impedindo violações de chave primária. Em seguida, realiza a inserção coordenada nas tabelas de atividade e de parceiro\footnote{A Stored Procedure foi projetada com flexibilidade para suportar tanto atividades realizadas em conjunto com o setor externo quanto atividades estritamente internas. Através de um controle de fluxo condicional, a rotina verifica a presença dos dados do parceiro. Caso estes parâmetros sejam nulos, a procedure realiza apenas a persistência da atividade, garantindo que o sistema não obrigue a criação de vínculos de parceria inexistentes ou o preenchimento de dados 'fantasmas' para satisfazer restrições de código.}. Por fim, a procedure aplica regras de sanitização de dados, convertendo automaticamente os nomes da atividade e da empresa para caracteres maiúsculos, garantindo a padronização textual no banco independentemente da formatação enviada pela aplicação cliente.